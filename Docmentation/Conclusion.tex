\documentclass{article}

\setlength{\parindent}{0pt}
\usepackage{cite}
\usepackage{tikz}
\newcommand\independent{\protect\mathpalette{\protect\independenT}{\perp}}
\def\independenT#1#2{\mathrel{\rlap{$#1#2$}\mkern2mu{#1#2}}}
\begin{document}
\section{Conclusion}
\subsection{Further Work}
The most obvious extension to this project would be to implement the FCI algorithm described in the literature review.\\

Another improvement would be to increase the speeed of the conditional independence tests. One approach could be to implement the independence test in C and interface with it using Cython. C programs are much faster than those written in python so there would almost definitely be an improvement.\\

Other independence tests could also be written. These tests could be for the same kind of data as the $\chi^2$ or it could be a test for other data for example continuous data.\\
\end{document}